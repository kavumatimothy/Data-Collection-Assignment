\documentclass[a4paper,10pt]{article}
 \pagenumbering{roman}
\begin{document}
\title{THE CHALLENGES FACED BY STATIONARIES IN MAKERERE UNIVERSITY.}
\author{
KAVUMA TIMOTHY.\\
student number 2150706.\\
regNo  15/U/6100/PS
}


\maketitle
\newpage
\section{Acknowledgement}
This research was supported by Stationary Operators in makerere university. I thank  them for optune time they gave to me during the research  and the insight they provided to their challenges.

\tableofcontents
\newpage
\pagenumbering{arabic}
\section{Introduction}
An increasing demand of courseworks and handouts lead to the outbreak and increase on the number of stationaries in makerere. This has helped students to reduce on the costs of transportation to access basic services provided by stationaries which were located in city centers. Stationaries are increasing in makerere as their challenges increase.

\section{ Problem Statement}
The rise of complaints from unsatisfied students cleary shows that there is an uncertain performance among the stationary workers. 
\section{Main Objective}
To identify the challenges faceed by stationaries in makerere university.
\subsection{Other objectives}
To clearly indentify the nature of the challenges whether they are administrative or social.\\
To find out how these stationaries have tried to handle the problems affecting them.

\section{Hypotheses}
Ho: Statinaary challenges are independent of the social, demographic and economic factors among the university adminstration.
\section{Significace of the study.}
The study will provide suitable solutions to the challenges that affect stationaries in makerere.
\section{Scope of study}
my study was based on the challenges faced by stationries owners in makerere university.

\section{Methodology}
My main source of data was primary data that was collected using the following methods: Observation and personal interviews.\\
The varriables under study included the number of years the operators have worked in their respective locations, the challenges they face and the space they occupy.
\subsection{Research Methods}
\subsubsection{Observation}
This included capturing of images my android phonem, observing and recording the spaces they occupy in the respective locations. 
\subsubsection{Personal Interviews}
This involved interactions with the stationaries operators and capturing the responses.
\section{Conclusion}
The challenges faced by stationaries operators in makerere university are independent of the social and economic factors among the university administration. They iclude: Limited space, noise from students, competition among rival operators, unstable power supply, high rent charges. If these problems are taken care of the services provided by the stationaries in makerere will surely be improved. 
\begin{thebibliography}{1}
\bibitem{}{ Benfield J. A.  Szlemko W. J. (2006).Available:http://jrp.icaap.org/index.php/jrp/article/view/30/51}
\bibitem{}{Open Data Kit(ODK)[online].Available:opendatakit.org/}

\end{thebibliography}

\end{document}